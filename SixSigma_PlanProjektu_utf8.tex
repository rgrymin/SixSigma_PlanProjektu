
\documentclass[printmode, oneside]{pwr}
%opcje klasy dokumentu mgr.cls zostały opisane w dołączonej instrukcji

%poniżej deklaracje użycia pakietów, usun±ć to co jest niepotrzebne
\usepackage{polski}       %przydatne podczas składania dokumentów w
%j. polskim 
%\usepackage[polish]{babel} %alternatywnie do pakietu
%polski, wybrać jeden z nich
\usepackage[utf8]{inputenc} %kodowanie znaków, zależne od systemu
\usepackage[T1]{fontenc} %poprawne składanie polskich czcionek

%pakiety do grafiki
\usepackage{graphicx}
\usepackage{subfigure}
\usepackage{psfrag}

%pakiety dodające dużo dodatkowych poleceń matematycznych
\usepackage{amsmath}
\usepackage{amsfonts}

%pakiety wspomagające i poprawiające składanie tabel
\usepackage{supertabular}
\usepackage{array}
\usepackage{tabularx}
\usepackage{hhline}

%pakiet wypisuj±cy na marginesie etykiety równań i rysunków
%zdefiniowanych przez \label{}, chc±c wygenerować finaln± wersję
%dokumentu wystarczy usunąć poniższ± linię
\usepackage{showlabels}

%definicje własnych poleceń
\newcommand{\R}{I\!\!R} %symbol liczb rzeczywistych, działa tylko w
                        %trybie matematycznym
\newtheorem{theorem}{Twierdzenie}[section] %nowe otoczenie do
                                           %składania twierdzeń

%dane do złożenia strony tytułowej
\title{Six Sigma (TODO)}
\engtitle{\ }
\author{
\bigskip
\begin{minipage}{2in}
\vfil
\mbox{Radosław Grymin (180499)}\\
\mbox{Karol Wons (172222)}
\end{minipage}
}

\supervisor{dr inż. Andrzej Rusiecki}
%\guardian{dr hab. inż. Imię Nazwisko Prof. PWr, I-6} %nie używać
%je¶li opiekun jest t± sam± osob± co prowadz±cy pracę

\date{2013} %standardowo u dołu strony tytułowej umieszczany jest
%bież±cy rok, to polecenie pozwala wstawić dowolny rok

%poniżej jest lista kierunków i specjalności na wydziale elektroniki,
%należy wybrać wła¶ciwe lub dopisać jeśli nie ma odpowiednich
\field{Automatyka i Robotyka (AIR)}
\specialisation{Technologie Informacyjne\\ w Systemach Automatyki (ART)}
\usepackage{url}
\begin{document}
\bibliographystyle{plabbrv} %tylko gdy używamy BibTeXa, ustawia polski
                            %styl bibliografii

\maketitle %polecenie generuj±ce stronę tytułową 

\tableofcontents %spis treści
 % wstawia pierwsze strony takie jak strona tytulowa,
                   % spis tresci, dedykacja
                   

% Tu zaczyna sie                  
\chapter{Opis projektu}
\section{Cel projektu}
  Celem projektu jest poznanie i zastosowanie statystycznych metod
wspomagających podejmowanie decyzji w metodyce zarządzania Lean Six Sigma.
Do badań użyto środowiska Minitab (skorzystano z licencji pracowniczej 
Radosława Grymina).

\section{Zarys projektu}
  Projekt będzie składał się z dwóch części: teoretycznego
opisu metodyki Six Sigma oraz
przeprowadzeniu doświadczeń na przykładowych danych.

Wdrażanie metodologii Six Sigma do działającego procesu
skupia się na iteracyjnym cyklu DMAIC (ang. \textit{Define, Measure, Analyze, Improve and Control}) bazującym na pomiarach działania procesu, statystycznej analizy i stałych 
popraw.

Doświadczenia przeprowadzane będą w programie Minitab wspierającym analizę statystyczną procesu.

\chapter{Harmonogram pracy}
	Prace nad projektem rozpoczynają się z dniem 23.10.2013 (początek tygodnia I).

\begin{itemize}
	\item \textbf{Tydzień I-II:} Zapoznanie się z tematyką Six Sigma, poszukiwanie literatury,
	zapoznanie się za środowiskiem Minitab,
	\item \textbf{Tydzień III-V:} Sporządzenie teoretycznego opisu metodyki Six Sigma.
	Konsultacje z prowadzącym.
	\item \textbf{Tydzień V-VII:} Wygenerowanie danych i dokonanie ich analizy.
	\item \textbf{Tydzień VIII:} Przedstawienie wyników prowadzącemu.
	\item \textbf{Tydzień VIII-X:} Ewentualne poprawki projektu.
\end{itemize}
	
\chapter{Wprowadzenie do Lean Six Sigma}
	Lean Six Sigma jest sprawdzoną metodologią usprawniania procesów
sterowaną przez potrzeby biznesowe oraz wymagania klientów.
Opiera się na danych pobieranych z procesu.

Lean Six Sigma skupia w sobie metodykę lean, która skupia się na redukcji wszelkiego
marnotrawstwa z procesu i usprawnianiu produktywności oraz Six Sigma, która skupia się 
na redukcji wariancji oraz defektów.

Termin Sigma odnosi się do statystycznej miary określającej jak daleko wyjście procesu zbacza
od jego celu  lub wymagań klienta.

\chapter{Bibliografia}
	\begin{enumerate}
\item ,,Six Sigma Statistics with Excel and Minitab'', Issa Bass, The McGraw-Hill Comapnies 2007
\item ,,SPC --- statystyczne sterowanie procesami produkcji'', T. Sałaciński, OWPW 2009 
\item ,,SPC --- statystyczne
sterowanie jakością'', \url{http://www.statsoft.pl/industries/spc.htm}

\end{enumerate}


\section{Czym jest Lean?}
	Lean jest systematyczną metodologią eliminacji marnotrawstwa lub eliminacji pracy niemającej wpływu
na wartość produktu (z perspektywy klienta, ang \emph{non-value added work}) 
oraz maksymalizacji wartości.
Lean zewnętrznie skupia się na maksymalizacji wartości z perspektywy klienta, a wewnętrznie na eliminacji marnotrawstwa oraz powiązanych kosztów procesu.

Lean jest 



% \appendix
% \chapter{Donec cursus nulla vitae pede}


% \addcontentsline{toc}{chapter}{Bibliografia} %utworzenie w
                                             %spisie treści pozycji
                                             %Bibliografia

% \bibliography{bibliografia} % wstawia bibliografię korzystaj±c z pliku
                            % bibliografia.bib - dotyczy BibTeXa,
                            % jeżeli nie korzystamy z BibTeXa należy
                            % użyć otoczenia thebibliography

%opcjonalnie może się tu pojawić spis rysunków i tabel
% \listoffigures
% \listoftables
\end{document}

