\documentclass[printmode, oneside]{pwr}
%opcje klasy dokumentu mgr.cls zostały opisane w dołączonej instrukcji

%poniżej deklaracje użycia pakietów, usun±ć to co jest niepotrzebne
\usepackage{polski}       %przydatne podczas składania dokumentów w
%j. polskim 
%\usepackage[polish]{babel} %alternatywnie do pakietu
%polski, wybrać jeden z nich
\usepackage[utf8]{inputenc} %kodowanie znaków, zależne od systemu
\usepackage[T1]{fontenc} %poprawne składanie polskich czcionek

%pakiety do grafiki
\usepackage{graphicx}
\usepackage{subfigure}
\usepackage{psfrag}

%pakiety dodające dużo dodatkowych poleceń matematycznych
\usepackage{amsmath}
\usepackage{amsfonts}

%pakiety wspomagające i poprawiające składanie tabel
\usepackage{supertabular}
\usepackage{array}
\usepackage{tabularx}
\usepackage{hhline}

%pakiet wypisuj±cy na marginesie etykiety równań i rysunków
%zdefiniowanych przez \label{}, chc±c wygenerować finaln± wersję
%dokumentu wystarczy usunąć poniższ± linię
\usepackage{showlabels}

%definicje własnych poleceń
\newcommand{\R}{I\!\!R} %symbol liczb rzeczywistych, działa tylko w
                        %trybie matematycznym
\newtheorem{theorem}{Twierdzenie}[section] %nowe otoczenie do
                                           %składania twierdzeń

%dane do złożenia strony tytułowej
\title{Six Sigma (TODO)}
\engtitle{\ }
\author{
\bigskip
\begin{minipage}{2in}
\vfil
\mbox{Radosław Grymin (180499)}\\
\mbox{Karol Wons (172222)}
\end{minipage}
}

\supervisor{dr inż. Andrzej Rusiecki}
%\guardian{dr hab. inż. Imię Nazwisko Prof. PWr, I-6} %nie używać
%je¶li opiekun jest t± sam± osob± co prowadz±cy pracę

\date{2013} %standardowo u dołu strony tytułowej umieszczany jest
%bież±cy rok, to polecenie pozwala wstawić dowolny rok

%poniżej jest lista kierunków i specjalności na wydziale elektroniki,
%należy wybrać wła¶ciwe lub dopisać jeśli nie ma odpowiednich
\field{Automatyka i Robotyka (AIR)}
\specialisation{Technologie Informacyjne\\ w Systemach Automatyki (ART)}