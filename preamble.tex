\documentclass[printmode, oneside]{pwr}

\usepackage{polski}       

\usepackage[utf8]{inputenc} % kodowanie znaków, zależne od systemu
\usepackage[T1]{fontenc}    % poprawne składanie polskich czcionek

% pakiety do grafiki
\usepackage{graphicx}
\usepackage{subfigure}
\usepackage{psfrag}

% pakiety dodające dużo dodatkowych poleceń matematycznych
\usepackage{amsmath}
\usepackage{amsfonts}

% pakiety wspomagające i poprawiające składanie tabel
\usepackage{supertabular}
\usepackage{array}
\usepackage{tabularx}
\usepackage{hhline}

% pakiet wypisujacy na marginesie etykiety równań i rysunków
% zdefiniowanych przez \label{}, chcac wygenerować finalna wersję
% dokumentu wystarczy usunąć poniższa linię
\usepackage{showlabels}

% definicje własnych poleceń
\newcommand{\R}{I\!\!R} % symbol liczb rzeczywistych, działa tylko w
                        % trybie matematycznym
\newtheorem{theorem}{Twierdzenie}[section] % nowe otoczenie do
                                           % składania twierdzeń

%dane do złożenia strony tytułowej
\title{Six Sigma w statystycznym sterowaniu procesami}
\engtitle{\ }
\author{
  \bigskip
  \begin{minipage}{2in}
    \vfil
    \mbox{Radosław Grymin (180499)}\\
    \mbox{Karol Wons (172222)}
  \end{minipage}
}

\supervisor{dr inż. Andrzej Rusiecki}

\date{2013} 

\field{Automatyka i Robotyka (AIR)}

\specialisation{
  Technologie Informacyjne\\
  w Systemach Automatyki (ART)
}
